%%%
% MẪU BÁO CÁO (REPORT) ĐƯỢC ĐỊNH DẠNG CƠ BẢN
% LIÊN HỆ TRỰC TIẾP NGƯỜI BIÊN SOẠN KHI CẦN HỖ TRỢ VỀ THAY ĐỔI TRONG GÓI LỆNH
% Huỳnh Phước Trường - Email: hphuoctruong.hcmue@gmail.com
%%%

% ĐỊNH DẠNG TÀI LIỆU
\documentclass[a4paper,12pt,twosided]{report}

% NGÔN NGỮ
\usepackage[vietnamese]{babel}
\usepackage[T1]{fontenc} 
\usepackage[utf8]{inputenc}

% TUỲ CHỈNH LỀ VĂN BẢN
\usepackage[a4paper,top=28mm,bottom=28mm,left=28mm,right=28mm,
headsep=6mm,footskip=5mm,includeheadfoot]{geometry}
\parskip = 10pt

%----------------------------------------------------------------------------------------
%	MỘT SỐ GÓI LỆNH CƠ BẢN
%----------------------------------------------------------------------------------------
\usepackage{tikz}
\usetikzlibrary{calc}
\usepackage{eso-pic}
\usepackage{amsmath,amsfonts,amssymb,commath,mathtools,enumerate,amsthm}
\usepackage{titlesec}
\usepackage{lipsum}
\usepackage{fancyhdr}


%----------------------------------------------------------------------------------------
%	GÓI LỆNH VẼ HÌNH
%----------------------------------------------------------------------------------------
\usepackage{pgf,tikz,pgfplots}
\pgfplotsset{compat=1.15}
\usepackage{mathrsfs}
\usetikzlibrary{arrows}
\usetikzlibrary{calc}

%----------------------------------------------------------------------------------------
%	GÓI MÀU
%----------------------------------------------------------------------------------------
\usepackage{color}
\definecolor{darkblue}{RGB}{0,0,170}
\definecolor{brickred}{RGB}{200,0,0}
\usepackage[hyperindex]{hyperref}
\hypersetup{colorlinks=true, linkcolor=darkblue, citecolor=brickred}



%----------------------------------------------------------------------------------------
%	MÔI TRƯỜNG ĐỊNH LÝ, ĐỊNH NGHĨA
%----------------------------------------------------------------------------------------
\makeatletter
\def\th@plain{%
  \thm@notefont{}% same as heading font
  \slshape % body font
}
\def\th@definition{%
  \thm@notefont{}% same as heading font
  \normalfont % body font
}
\makeatother
\theoremstyle{plain}
\newcounter{dummy} 
\numberwithin{dummy}{section}
\newtheorem{theorem}[dummy]{Định lý}
\newtheorem{definition}[dummy]{Định nghĩa}
\newtheorem{property}[dummy]{Tính chất}
\newtheorem{lemma}[dummy]{Bổ đề}
\newtheorem{proposition}[dummy]{Mệnh đề}
\newtheorem{corollary}[dummy]{Hệ quả}
\newtheorem{example}[dummy]{Ví dụ}
\newtheorem{remark}[dummy]{Nhận xét}
\addto\captionsvietnamese{\renewcommand\proofname{\bf Chứng minh}}
\renewcommand\qedsymbol{$\blacksquare$} % Hình vuông màu đen cuối chứng minh.


%----------------------------------------------------------------------------------------
%	THIẾT LẬP CÁCH ĐÁNH SỐ, KIỂU ĐÁNH SỐ MỤC VÀ CHƯƠNG
%----------------------------------------------------------------------------------------
\titleformat{\section}
{\normalfont\Large \sffamily\bfseries}{\arabic{section}.}{0.5em}{}

%----------------------------------------------------------------------------------------
%	THIẾT LẬP ĐÁNH SỐ TRANG
%----------------------------------------------------------------------------------------
\fancyhf{}
%\renewcommand{\headrulewidth}{0pt}  % Xóa dòng kẻ ở header
\fancyhead[C]{\thepage}
\pagestyle{fancy}

\fancypagestyle{plain}{%
\fancyhf{} % clear all header and footer fields
\fancyhead[C]{\thepage} % except the center
}
 % Phần thiết lập chung, mở file để thay đổi các thiết lập


\begin{document}
\overfullrule=5mm % Xác định vị trí cuối trang có văn bản bị quá dòng

% TRANG BÌA
\begin{titlepage}
\begin{center}
{\Large \scshape
Trường Đại học Sư phạm Thành phố Hồ Chí Minh\\
Khoa Toán - Tin học\\}

\rule{6cm}{1pt}

\vspace{1cm} % Logos của trường và khoa

\includegraphics[scale=0.2]{logo/logo.png}\hspace{1cm}
\includegraphics[scale=0.5]{logo/logokhoa.png}
\end{center}

\vspace{1cm}

\begin{center}
\begin{minipage}{\textwidth}
{\Large \bf BÁO CÁO KẾT THÚC MÔN HỌC}\\

\rule{\textwidth}{0.8pt}
\begin{center}
{\large Thêm tựa bài tại đây}
\end{center}
\rule{\textwidth}{0.8pt}
\end{minipage}
\end{center}

\vspace{1cm}

\begin{table}[h]
\begin{tabular}{rrl}
\hspace{3 cm} & Giảng viên hướng dẫn: &TS. ABC\\
& Sinh viên thực hiện: & Huỳnh Phước Trường - K40.101.161\\
& & Nguyễn Phước Toàn - K40.101.153  \\
& & Nguyễn Phước Toàn - K40.101.153  \\
& & Nguyễn Phước Toàn - K40.101.153  \\
\end{tabular}
\end{table}

\vspace{2cm} % Lùi xuống dòng 2cm

\begin{center}
{\large \scshape TP.HCM - Năm 2021}
\end{center}
\end{titlepage}

\tableofcontents
\newpage


% Chương 1.
\chapter{Chương đầu tiên}

\section{Bài học đầu tiên}

\begin{theorem}Cho các số dương $a,b,c$ thỏa mãn $a+b+c = 3$. Chứng minh rằng
$$a^2+b^2+c^2 \ge 3.$$
\end{theorem}

\begin{proof} 
\end{proof}

Đoạn sau để thử việc cảnh báo quá dòng.
$$ \int_{\Omega} f(x) dx = a_1+a_2+a_3+a_4+ a_1+a_2+a_3+a_4+ a_1+a_2+a_3+a_4+ a_1+a_2+a_4+ 
\dfrac{1}{a_1+a_2+a_3+a_4}.$$

\cite{Reference1}.


% Chương 2.
\chapter{Chương 2}



\addcontentsline{toc}{chapter}{Tài liệu tham khảo}
\bibliographystyle{IEEEtran}
\bibliography{mybib}

\end{document}

