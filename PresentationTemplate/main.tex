%%%%%%%%%%%%%%%%%%%%%%%%%%%%%%%%%%%%%%%%%
% Bachelor's/Master's Thesis - Ho Chi Minh City University of Education
% LaTeX Template
%
% Template được làm dựa trên phiên bản của:
% Steve Gunn (http://users.ecs.soton.ac.uk/srg/softwaretools/document/templates/)
% Sunil Patel (http://www.sunilpatel.co.uk/thesis-template/)
% Thực hiện: Nhóm Sinh viên K44:
% Nguyễn Thị Minh Hằng
% Nguyễn Thu Hà
% Đặng Thị Thục Quyên
% Hoàng Nam Phương
% Mọi thắc mắc xin liên hệ về email: hphuoctruong.hcmue@gmail.com
%%%%%%%%%%%%%%%%%%%%%%%%%%%%%%%%%%%%%%%%%


% CÁC GÓI VÀ ĐỊNH DẠNG TÀI LIỆU
\documentclass[12pt, % cỡ chữ mặc định, các lựa chọn khác: 10pt, 11pt, 12pt
twoside, % lựa chọn khác: oneside (in một mặt)
singlespacing, % giãn cách dòng, các lựa khác: onehalfspacing, doublespacing
%nolistspacing, % Nếu tài liệu đang dùng cách dãn dòng onehalfspacing hay doublespacing, sử dụng lệnh này để dãn dòng danh sách theo kiểu single
%liststotoc, % Sử dụng lệnh này để thêm danh sách hình ảnh / bảng /v.v. vào mục lục
% toctotoc, % Dùng lệnh này để thêm "Mục lục" vào mục lục
%parskip, % Dùng lệnh này để thêm khoảng cách giữa các đoạn
%nohyperref, % Dùng lệnh này để tắt gói hyperref package
headsepline, % Sự dụng lệnh này để có đường gạch dưới header
%chapterinoneline, % Dùng lệnh này để đặt tiêu đề chương và số thứ tự chương trên một dòng
%consistentlayout, % Sử dụng lệnh này để thay đổi layout của các trang lời cam đoan, tóm tắt và lời biết ơn để khớp với layout mặc định
]{ThesisHCMUE} 


% NGÔN NGỮ
\usepackage[utf8]{vietnam} 
\usepackage[T1]{fontenc} 
\usepackage[main =  vietnamese, english]{babel} % Ngôn ngữ tiếng Việt nếu main = vietnamese, tiếng Anh nếu main = english.
%\usepackage{times} %Chọn font chữ khác với font chữ mặc định
%----------------------------------------------------------------------------------------
%	MỘT SỐ GÓI LỆNH CƠ BẢN
%----------------------------------------------------------------------------------------
\usepackage{tikz}
\usetikzlibrary{calc}
\usepackage{eso-pic}
\usepackage{amsmath,amsfonts,amssymb,commath,mathtools,enumerate,amsthm}
\usepackage{titlesec}
\usepackage{lipsum}
\usepackage{fancyhdr}


%----------------------------------------------------------------------------------------
%	GÓI LỆNH VẼ HÌNH
%----------------------------------------------------------------------------------------
\usepackage{pgf,tikz,pgfplots}
\pgfplotsset{compat=1.15}
\usepackage{mathrsfs}
\usetikzlibrary{arrows}
\usetikzlibrary{calc}

%----------------------------------------------------------------------------------------
%	GÓI MÀU
%----------------------------------------------------------------------------------------
\usepackage{color}
\definecolor{darkblue}{RGB}{0,0,170}
\definecolor{brickred}{RGB}{200,0,0}
\usepackage[hyperindex]{hyperref}
\hypersetup{colorlinks=true, linkcolor=darkblue, citecolor=brickred}



%----------------------------------------------------------------------------------------
%	MÔI TRƯỜNG ĐỊNH LÝ, ĐỊNH NGHĨA
%----------------------------------------------------------------------------------------
\makeatletter
\def\th@plain{%
  \thm@notefont{}% same as heading font
  \slshape % body font
}
\def\th@definition{%
  \thm@notefont{}% same as heading font
  \normalfont % body font
}
\makeatother
\theoremstyle{plain}
\newcounter{dummy} 
\numberwithin{dummy}{section}
\newtheorem{theorem}[dummy]{Định lý}
\newtheorem{definition}[dummy]{Định nghĩa}
\newtheorem{property}[dummy]{Tính chất}
\newtheorem{lemma}[dummy]{Bổ đề}
\newtheorem{proposition}[dummy]{Mệnh đề}
\newtheorem{corollary}[dummy]{Hệ quả}
\newtheorem{example}[dummy]{Ví dụ}
\newtheorem{remark}[dummy]{Nhận xét}
\addto\captionsvietnamese{\renewcommand\proofname{\bf Chứng minh}}
\renewcommand\qedsymbol{$\blacksquare$} % Hình vuông màu đen cuối chứng minh.


%----------------------------------------------------------------------------------------
%	THIẾT LẬP CÁCH ĐÁNH SỐ, KIỂU ĐÁNH SỐ MỤC VÀ CHƯƠNG
%----------------------------------------------------------------------------------------
\titleformat{\section}
{\normalfont\Large \sffamily\bfseries}{\arabic{section}.}{0.5em}{}

%----------------------------------------------------------------------------------------
%	THIẾT LẬP ĐÁNH SỐ TRANG
%----------------------------------------------------------------------------------------
\fancyhf{}
%\renewcommand{\headrulewidth}{0pt}  % Xóa dòng kẻ ở header
\fancyhead[C]{\thepage}
\pagestyle{fancy}

\fancypagestyle{plain}{%
\fancyhf{} % clear all header and footer fields
\fancyhead[C]{\thepage} % except the center
}
 % Phần thiết lập chung


% TUỲ CHỈNH LỀ VĂN BẢN
\usepackage[a4paper,top=30mm,bottom=30mm,left=30mm,right=25mm,headsep=6mm,footskip=5mm,includeheadfoot]{geometry}
% \numberwithin{dummy}{chapter} % Đánh số định lý theo chương
\parskip = 10pt


% THÔNG TIN LUẬN VĂN
\thesistitle{HÀM ĐIỀU HOÀ}
% Tên luận văn. In nó ở bất kỳ đâu bằng lệnh \ttitle
\author{NGUYỄN VĂN A.} 								
% Tên người thực hiện. In bằng lệnh \authorname
\supervisor{TS. PHẠM THỊ B.} 						
% Tên người hướng dẫn. In bằng lệnh \supname
\degree{ĐẠI HỌC}
% Trình độ. In bằng lệnh \degreename
\typethesis{KHOÁ LUẬN TỐT NGHIỆP ĐẠI HỌC} 			
% Loại luận văn. In bằng lệnh \TypeThesis
\subject{TOÁN HỌC} 										
% Môn. In bằng lệnh \subjectname
\field{TOÁN ỨNG DỤNG}
% Chuyên ngành.	In bằng lệnh \fieldname
\keywords{} % Từ khoá của luận văn,  hiện tại chưa được dùng ở bất cứ đâu trong văn bản, in nó bằng lệnh \keywordnames
\university{TRƯỜNG ĐẠI HỌC SƯ PHẠM THÀNH PHỐ HỒ CHÍ MINH} 
% Tên trường đại học và website trường. được dùng trong trang bìa, in bằng lệnh \univname
\department{KHOA TOÁN - TIN HỌC} 
% Tên khoa và website, được dùng ở trang bìa, in bằng lệnh \deptname

% THÔNG TIN TRÍCH DẪN
\AtBeginDocument{
\hypersetup{pdftitle=\ttitle}
\hypersetup{pdfauthor=\authorname}
\hypersetup{pdfkeywords=\keywordnames}}
%\usepackage[backend=bibtex,style=authoryear,natbib=true]{biblatex} % Kiểu trích dẫn
%\addbibresource{mybib.bib} % File tài liệu tham khảo
\usepackage[autostyle=true]{csquotes}


\begin{document}
\frontmatter % Dùng kiểu đánh số La Mã (i, ii, iii, iv...) cho các trang trước phần nội dung chính.

\pagestyle{plain}

%------------------------------------------------
%	TRANG BÌA CHÍNH - BÌA PHỤ
%   Đây là hai trang bìa chính và phụ. Nếu không cần thay đổi gì ở hai trang này, có thể bỏ qua và đi vào phần chính.
%------------------------------------------------

%------------------------------------------------
%   TRANG BÌA CHÍNH
%------------------------------------------------
\begin{titlepage}
\border{1in} % Vị trí đặt khung của trang bìa. Bỏ dòng này đi nếu không cần khung.

\begin{center}
\boldfont{16}{\ministry}\\ % Bộ Giáo dục và Đào tạo

\boldfont{16}{\univdefault}\\ % Tên trường, có thể thay bằng \univname điền ở phía trên phần khai báo

\boldfont{16}{\deptdefault}\\ % Tên khoa, có thể thay bằng \deptname điền ở phía trên phần khai báo
\rule{6cm}{1pt}

\vspace{1cm} % Logos của trường và khoa
\includegraphics[scale=0.2]{logo/logo.png}\hspace{1cm}
\includegraphics[scale=0.5]{logo/logokhoa.png}

\vspace{2cm}
\boldfont{20}{\authorname} % Tên tác giả

\vspace{3cm}
\boldfont{24}{\ttitle} % Tên đề tài

\vspace{3cm}
\boldfont{16}{ \TypeThesis} % Loại luận văn
\end{center}
\vfill

\begin{center}
\boldfont{16}{\city - \the\year} % TP.HCM - Năm ...
\end{center}
\end{titlepage}

%------------------------------------------------
%	TRANG BÌA PHỤ
%------------------------------------------------
\begin{titlepage}
\border{1in}

\begin{center}
\boldfont{16}{\ministry}\\

\boldfont{16}{\univdefault}\\

\boldfont{16}{\deptdefault}\\
\rule{6cm}{1pt}


\vspace{3cm}
\boldfont{20}{\authorname}


\vspace{3cm}
\boldfont{24}{\ttitle}

\end{center}

\vspace{3cm}
\basicfont{16pt}{\color{blue} \fieldofstudy: \fieldname}
%\basicfont{16pt}{\color{blue} Mã số:}

\begin{center}
\vspace{2cm}
\boldfont{16}{\TypeThesis} 
\end{center}

\vspace{1.5cm}
\begin{flushright}
\basicfont{16pt}{\supervisorname: \\ \supname}
\end{flushright}


\vfill

\begin{center}
\boldfont{16}{\city - \the\year}
\end{center}
\end{titlepage}

%------------------------------------------------
%	TRANG CAM ĐOAN
%------------------------------------------------

\begin{declaration}
\addchaptertocentry{\authorshipname} % Thêm trang này vào mục lục
I hereby certify that the thesis I am submitting is entirely my own original work except where otherwise indicated. I am aware of the University's regulations concerning plagiarism, including those regulations concerning disciplinary actions that may result from plagiarism. Any use of the works of any other author, in any form, is properly acknowledged at their point of use.

Tôi xin cam đoan rằng luận văn này được chính tôi thực hiện, các thông tin trích dẫn trong luận văn này đều được chỉ rõ nguồn gốc.
\end{declaration}
\cleardoublepage

%----------------------------------------------------------------------------------------
%	TRANG TRÍCH DẪN
%----------------------------------------------------------------------------------------

\vspace*{0.2\textheight}

\noindent{\itshape If people do not believe that mathematics is simple, it is only $12345678$ because they do not realize how complicated life is.}\bigbreak

\hfill John von Neumann

%----------------------------------------------------------------------------------------
%	TRANG TÓM TẮT
%----------------------------------------------------------------------------------------

\begin{abstract}
\addchaptertocentry{\abstractname} 
\thispagestyle{empty}  % Bỏ dấu % để tạo trang trắng, không hiện số trang.
The thesis abstract is written here.
\end{abstract}

%----------------------------------------------------------------------------------------
%	TRANG LỜI CẢM ƠN
%----------------------------------------------------------------------------------------

\begin{acknowledgements}
\addchaptertocentry{\acknowledgementname} % Thêm trang này vào mục lục
I would like to express my deepest gratitude to my supervisor, Nguyễn Phư \ldots
\end{acknowledgements}

%----------------------------------------------------------------------------------------
%	TRANG LỜI GỬI TẶNG
%----------------------------------------------------------------------------------------

\dedicatory{Gửi đến gia đình tôi \ldots} 

%----------------------------------------------------------------------------------------
%	MỤC LỤC
%----------------------------------------------------------------------------------------

\tableofcontents 
%\listoffigures % Danh sách hình ảnh - Nếu dùng thì bỏ dấu % ở đầu
%\listoftables  % Danh sách bảng biểu - Nếu dùng thì bỏ dấu % ở đầu

%----------------------------------------------------------------------------------------
%	TRANG TỪ VIẾT TẮT
%----------------------------------------------------------------------------------------

% Nếu không dùng có thể xoá toàn bộ trang này đi.

\begin{abbreviations}{ll} % Bao gồm danh sách từ viết tắt (một cái bảng có hai cột)

\textbf{LAH} & \textbf{L}ist \textbf{A}bbreviations \textbf{H}ere\\
\textbf{WSF} & \textbf{W}hat (it) \textbf{S}tands \textbf{F}or\\

\end{abbreviations}

%----------------------------------------------------------------------------------------
%	TRANG KÝ HIỆU 
%----------------------------------------------------------------------------------------

% Nếu không dùng có thể xoá toàn bộ trang này đi.

\begin{symbols}{ll} % Bao gồm danh sách các ký hiệu sử dụng (một bảng gồm hai cột)

$\mathbb{N}$ & Tập hợp các số tự nhiên \\
$\mathbb{R}^*$ & Tập hợp các số thực khác $0$ \\
\addlinespace 
$L^p(\Omega)$ & Không gian các hàm khả tích bậc $p$ trên $\Omega$
\end{symbols}

%----------------------------------------------------------------------------------------
%	NỘI DUNG LUẬN VĂN - CÁC CHƯƠNG
%----------------------------------------------------------------------------------------

\mainmatter
\pagestyle{thesis}
\overfullrule=5mm

% Các file luận văn được để trong folder Chapters. Cần dùng file nào thì include tương ứng.

% Chapter 1
\chapter{Giới thiệu mẫu luận văn}
\label{chap_huongdan} % Gán nhãn, sẽ được hướng dẫn chi tiết phía dưới.

%----------------------------------------------------------------------------------------
\newcommand{\keyword}[1]{\textbf{#1}}
\newcommand{\tabhead}[1]{\textbf{#1}}
\newcommand{\code}[1]{\texttt{#1}}
\newcommand{\file}[1]{\texttt{\bfseries#1}}
\newcommand{\option}[1]{\texttt{\itshape#1}}
%----------------------------------------------------------------------------------------

Chúng tôi xin giới thiệu việc sử dụng mẫu luận văn ở chương này. Cụ thể, ngoài những nội dung thường có ở một file .tex thông thường, mẫu luận văn này có những nội dung riêng, phục vụ cho trường Đại học Sư phạm TP.HCM nói chung và khoa Toán - Tin học ở trường nói riêng.

\section{Thiết lập chung}
File chính chứa thông tin định dạng luận văn là {\tt ThesisHCMUE.cls}. Ngoài ra, một số thiết lập cơ bản được chứa trong file {\tt settings.tex} và phần đầu trang {\tt main.tex}.

Những thông tin chính của luận văn (tên tác giả, tên luận văn, tên người hướng dẫn, ... ) có thể được thay trực tiếp tại các vị trí tương ứng trong file {\tt main.tex}. Vị trí của các thông tin đã được chú thích vắn tắt trong file trên (từ dòng 46 đến dòng 65, nếu chưa có thay đổi nào ở file đó).

Để biên dịch luận văn, ta có thể dùng trực tiếp file {\tt main.tex}, hoặc thiết lập file {\tt main.tex} làm file chính (Master document).

Ngôn ngữ mặc định của luận văn là tiếng Việt. Ngoài ra, mẫu luận văn có bao gồm việc chuyển đổi sang tiếng Anh. Để thay đổi các mục sang tiếng Anh (ở trang bìa, các Định nghĩa, Định lý, ..., trang Tóm tắt, trang Mục lục, sẽ được nói rõ hơn ở các phần sau), người sử dụng có thể thay dòng 35 trong file {\tt main.tex}.

\begin{verbatim}
    \usepackage[main =  vietnamese, english]{babel}
\end{verbatim}
thành
\begin{verbatim}
    \usepackage[vietnamese, main =  english]{babel}
\end{verbatim}

\section{Môi trường định nghĩa, định lý, bổ đề, ...}
Trong gói luận văn đã thiết lập sẵn các môi trường Định nghĩa, Bổ đề, Tính chất, ... Khi cần đưa các môi trường này vào trong văn bản, ta sử dụng các lệnh như trong mẫu dưới đây. 
\begin{verbatim}
\begin{}
\end{}
\end{verbatim}
Trong dấu $\{ \}$ chúng ta điền tên môi trường cần sử dụng vào. Gói luận văn này gồm các lệnh sau:
\begin{table}[!htp]
\centering
\begin{tabular}{|l|l|}
\hline
\textbf{Môi trường cần sử dụng} & \textbf{Tên lệnh tương ứng} \\
\hline
Định nghĩa & definition  \\
\hline
Định lý & theorem  \\
\hline
Hệ quả & corollary \\
\hline
Tính chất & property  \\
\hline
Ví dụ & example  \\
\hline
Mệnh đề & proposition  \\
\hline
Nhận xét & remark \\
\hline
Bổ đề & lemma \\
\hline
\end{tabular}
\end{table}

Sau đây là một số ví dụ.

\begin{theorem}[Định lý hàm số cosine]
Với tam giác $ABC$ tuỳ ý ta luôn có
\begin{equation} \label{eq_dinhlycos} % Dán nhãn cho phương trình để gọi lại phía sau.
    AB^2+AC^2-2AB \cdot AC \cdot \cos A = BC^2.
\end{equation}
\end{theorem}

\begin{corollary}
Tam giác $ABC$ vuông tại $A$ khi và chỉ khi $AB^2+AC^2=BC^2$.
\end{corollary}

\begin{proof}
Trong phương trình \eqref{eq_dinhlycos}, ta cho $\angle A = 90^{\circ}$.
\end{proof}

\begin{example}
Cho tam giác $ABC$ có $AB=AC=1$ và $BC=\sqrt{2}$. Khi đó tam giác $ABC$ vuông cân tại $A$.
\end{example}



%----------------------------------------------------------------------------------------
\section{Vẽ hình, chèn hình, chèn bảng}
\subsection{Chèn hình}
Khi muốn chèn hình, chúng ta sẽ dùng lệnh
\begin{verbatim} 
\begin{figure}[vị trí đặt ảnh]
    \includegraphics[scale=...]{Figures/Tên file hình}
    \caption{Tên hình, mô tả hình}
    \label{Tên nhãn}
\end{figure}
\end{verbatim}
Trong đó,
\begin{itemize}
    \item ta phải bỏ tất cả file hình vào folder \textit{Figures}.
    \item \textit{scale} là lệnh để tùy chỉnh kích thước của hình.
    \item \textit{caption} là tên hình.
    \item \textit{label} là nhãn, tên dùng khi muốn gọi lại hình ảnh.
    \item Các vị trí ưu tiên để đặt ảnh: h(here, đặt ở vị trí hiện tại); t(top, đặt ở vị trí trên cùng trang hiển thị); b(bottom, đặt ở cuối trang hiển thị)...
\end{itemize}

Sau đây là một ví dụ.

\begin{figure}[!ht]
\centering
\includegraphics[scale=0.2]{Figures/Electron}
\caption[An Electron]{An electron (artist's impression).}
\label{fig:Electron1}
\end{figure}

Tôi muốn gọi lại Hình \ref{fig:Electron1} ở trang \pageref{fig:Electron1}.

\subsection{Chèn bảng}
Khi muốn tạo bảng trong LaTeX, chúng ta sử dụng môi trường \textbf{table} như sau 
\begin{verbatim} 
\begin{table}[Vị trí muốn đặt bảng]
\caption{Tên bảng}
\label{Tên nhãn}
\centering
    \begin{tabular}{Tuỳ chọn}
    \end{tabular}
\end{table}
\end{verbatim}
Sau đây là một ví dụ.

\begin{table}[!htp]
\caption{Tên của các môi trường}
\label{tab: Bang1}
\centering
\begin{tabular}{|l|l|}
\hline
\textbf{Môi trường cần sử dụng} & \textbf{Tên điền vào} \\
\hline
Định nghĩa & definition  \\
\hline
Định lý & theorem  \\
\hline
Hệ quả & corollary \\
\hline
Tính chất & property  \\
\hline
Ví dụ & example  \\
\hline
Mệnh đề & proposition  \\
\hline
Nhận xét & remark \\
\hline
Bổ đề & lemma \\
\hline
\end{tabular}
\end{table}

Nhìn vào bảng \ref{tab: Bang1}, ta thấy ...

\section{Cảnh báo về canh lề}
Luận văn đã thiết lập lề theo mẫu được quy định tại trang web của Phòng Sau đại học Trường Đại học Sư phạm TP.HCM (lề trái 30 mm, lề phải 25 mm, lề trên 30 mm, lề dưới 30 mm, khổ giấy A4). Khi có một dòng vượt ngoài lề giấy đã quy định (có thể gây ra lỗi khi in luận văn), file sẽ có thông báo bằng một ô vuông ở cuối dòng. Ví dụ
$$ \dfrac{a^2}{b+c+2a} + \dfrac{a^2}{b+c+2a}  + \dfrac{a^2}{b+c+2a}  \le \dfrac{a^2}{b+c+2a}  + \dfrac{a^2}{b+c+2a}  + \dfrac{a^2}{b+c+2a} + 3abc+ 3(ab+bc+ca).$$

\section{Trích dẫn}
Mẫu luận văn này được thiết lập trích dẫn tài liệu theo kiểu IEEE, được quy định tại trang web của Phòng Sau đại học Trường Đại học Sư phạm TP.HCM. Chi tiết có thể xem tại \\
\url{https://drive.google.com/file/d/1SsiUbQtHs6tCtWbiVkq0Ry38baFlbxQd/view}.

Các tài liệu trích dẫn được lưu trong file {\tt mybib.bib}. Để trích dẫn tài liệu, ta dùng lệnh
\begin{verbatim} 
\cite[tuỳ chọn]{tên tài liệu}
\end{verbatim}
Ví dụ: \cite{Reference1}, \cite[Theorem 1.1]{Reference2}.

Ngoài ra, người dùng có thể tự thêm danh sách tài liệu ``thủ công'' mà không cần dùng file {\tt mybib.bib}.





\chapter{Quy định trình bày luận văn} \label{chap_quydinh}

Nội dung chi tiết có thể xem tại
\url{http://hcmup.edu.vn/index.php?option=com_content&view=article&id=23554%3Abieu-mau-luan-van&catid=5914%3Aths-biu-mu-&Itemid=9980&lang=en&site=14}
 

%----------------------------------------------------------------------------------------
%	NỘI DUNG LUẬN VĂN - PHỤ LỤC
%----------------------------------------------------------------------------------------

\appendix %PHỤ LỤC

% Bao gồm phần phụ lục của luận văn, mỗi phần phụ lục là các file riêng từ folder Appendices

% Appendix A

\chapter{Phụ lục}
\label{append_phuluc}

Đây là nơi thêm vào phụ lục cho luận văn.

%\include{Appendices/AppendixB}

%----------------------------------------------------------------------------------------
%	TÀI LIỆU THAM KHẢO
%----------------------------------------------------------------------------------------

\newpage
\addcontentsline{toc}{chapter}{References}
\bibliographystyle{IEEEtran}
\bibliography{mybib}

\end{document}  